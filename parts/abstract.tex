\begin{abstract}

The digital revolution has not only left its traces in the world of medicine, aeronautics or telecommunications, but also in the realm of electronic music production. While physical music synthesizers once produced sound through analog circuitry, they can now be implemented virtually --- in software --- using computer programming languages such as C++. This thesis aims to describe how the creation of such a software synthesizer can be achieved. The first chapter introduces rudimentary concepts of digital audio processing and explains how digital sound differs from its analog counterpart. Subsequently, chapter two discusses the generation of sound in software --- using either mathematical calculation or Additive Synthesis, outlines how digital waveforms can be efficiently stored in a Wavetable and, lastly, touches upon the concept of Noise. Chapter three investigates how modulation sources such as Audio Envelopes or Low Frequency Oscillators can be used to modify amplitude, timbre and other properties of an audio signal.  It also proposes a means of flexible modulation of a variable number of parameters in a synthesis system by an equally unfixed number of modulation sources. The last chapter explains how Frequency Modulation Synthesis can be used to synthesize audio signals and thereby produce entirely new, electronic sounds. Embedded into all chapters are numerous C++ computer programs as well as images, diagrams and charts to practically illustrate the theoretical concepts of digital audio processing at hand.

\end{abstract}
