\chapter*{Conclusion}
\addcontentsline{toc}{chapter}{Conclusion}

This thesis has shown the most important steps of how the C++ programming language can be used to create a digital synthesis system. It has used available knowledge sources off and online to discuss the theoretical concepts behind digital audio processing and then gave practical, tested examples of how these concepts can be implemented in C++. \vspace{\baselineskip}

The first chapter outlined the basic knowledge required to discuss digital sound, and hinted at the differences between sound in the analog and the digital realm. Firstly, it was postulated that in the digital world, sound is not a continuous, but a a discrete, signal, composed of many individual recordings or \emph{samples}. Chapter One also explained that the rate at which these samples are recorded from a continuous signal is called the \emph{sample rate}. Moreover, it was discussed that the maximum frequency of a signal to be sampled must be less than or equal to one half of the sample rate -- the \emph{Nyquist Limit} -- to ensure \emph{proper sampling}. Lastly, a common phenomenon associated with sample rates, \emph{Aliasing}, which occurs when a signal is sampled improperly and which results in a misrepresentation of the signal, when \emph{quantized}, was also examined. \vspace{\baselineskip}

The subsequent chapter dealt with the generation of digital sound. It explained that audio waveforms could be created either mathematically, yielding \emph{exact} waveforms, or via Additive Synthesis, producing more natural sounds. It was then shown how three popular waveforms --- square, sawtooth and triangle --- can be created with Additive Synthesis. Then, the concept of a \emph{Wavetable}, to efficiently store and play-back such waveforms, was also investigated. Lastly, a special form of sound, \emph{Noise}, was studied and its implementation discussed. \vspace{\baselineskip}

Chapter Three concerned itself with the modulation of sound. Two common means of modulation, Audio Envelopes and Low Frequency Oscillators (LFOs), were examined in detail. This chapter also gave a concrete example of how a modular dock of modulation sources, a so-called \emph{ModDock}, is a very efficient and intuitive method of making it possible to modulate a variable number of parameters in a synthesis system by a variable number of modulation sources. \vspace{\baselineskip}

The last chapter analyzed how many different sound signals can be \emph{synthesized} to give an entirely new signal. Popular methods of synthesis were mentioned, while one --- Frequency Modulation (FM) Synthesis --- was discussed in particular. Some important concepts concerning FM Synthesis, such as the $C:M$ ratio, FM algorithms, or the idea of sidebands, were examined. Followingly, the steps were shown of how FM Synthesis can be implemented in C++ and finally used in a real synthesis system. \vspace{\baselineskip}

\pagebreak

In summary, it can be said that this thesis showed the most essential steps of creating a digital synthesizer. Naturally, not all steps of the complete process were shown. Also, it should be mentioned that this thesis did not discuss cutting-edge research or recent findings that may be used in commercial synthesizers today, but tried to discuss approved and already implemented methods. However, enough theory and practical examples for anyone interested in creating his or her own digital synthesizer in C++, to begin the journey, was given.
