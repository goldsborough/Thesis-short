\begin{thebibliography}{9}
\addcontentsline{toc}{chapter}{Bibliography}

\bibitem{bsynth}

  Daniel R. Mitchell,

  \emph{BasicSynth: Creating a Music Synthesizer in Software}.

  Publisher: Author.

  1st Edition,

  2008.

\bibitem{dspguide}

  Steven W. Smith,

  \emph{The Scientist and Engineer's Guide to Digital Signal Processing}.

  California Technical Publishing,

  San Diego, California,

  2nd Edition,

  1999.

\bibitem{oxdic}

  Ed: Catherine Soanes and Angus Stevenson,

  \emph{Oxford Dictionary of English}.

  Oxford University Press,

  Oxford,

  2003.

\bibitem{musco}

  Phil Burk, Larry Polansky, Douglas Repetto, Mary Roberts and Dan Rockmore,

  \emph{Music and Computers: A Historical and Theoretical Approach}.

  2011

  \url{http://music.columbia.edu/cmc/MusicAndComputers/}

  Accessed: 22 December 2014.

\bibitem{sosfm1}

  Gordon Reid,

  \emph{Synth Secrets, Part 12: An Introduction To Frequency Modulation}.

  \url{http://www.soundonsound.com/sos/apr00/articles/synthsecrets.htm}

  Accessed: 30 December 2014.

\bibitem{sosfm2}

  Gordon Reid,

  \emph{Synth Secrets, Part 13: More On Frequency Modulation}.

  \url{http://www.soundonsound.com/sos/may00/articles/synth.htm}

  Accessed: 30 December 2014.

\bibitem{sfufm}

  \emph{Tutorial for Frequency Modulation Synthesis}.

  \url{http://www.sfu.ca/~truax/fmtut.html}

  Accessed: 30 December 2014.

\bibitem{samplerates}

  Justin Colletti,

  \emph{The Science of Sample Rates (When Higher Is Better --- And When It Isn't)}.

  2013.\\
  \url{http://www.trustmeimascientist.com/2013/02/04/the-science-of-sample-rates-when-higher-is-better-and-when-it-isnt/}

  Accessed: 17 December 2014.

\bibitem{hearing}

  John D. Cutnell and Kenneth W. Johnson,

  \emph{Physics}.

  Wiley,

  New York,

  4th Edition,

  1998.

\bibitem{dx7}

  Electronic Music Wiki,

  \emph{DX7}.

  \url{http://electronicmusic.wikia.com/wiki/DX7}

  Accessed: 4 January 2015.

\end{thebibliography}
