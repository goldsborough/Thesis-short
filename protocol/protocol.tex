% Developing a Digital Synthesizer in C++ --- Protocol

\documentclass[12pt]{report}

\usepackage[margin=1in, a4paper]{geometry}

\usepackage{colortbl}

\usepackage{graphicx}

\usepackage{newtxtext}

\setlength{\headheight}{15pt}

\setlength{\parindent}{0pt}

\newcommand{\parbreak}{\vspace{\baselineskip}}

\begin{document}

\chapter*{Protocol}

\textbf{Name of student}: Peter Goldsborough \parbreak

\textbf{Subject of thesis}: Developing a Digital Synthesizer in C++ \parbreak

\textbf{Name of supervisor}: Prof. Martin Kastner \parbreak

\textbf{Final length of thesis}: 63349 characters \parbreak

\textbf{Reasons for trespassing of character limit}: This thesis was written with the \LaTeX{} document preparation system, which has no built-in features to calculate either the word count nor the character count. The number of characters given above was determined by converting the \LaTeX{}-generated PDF file to a TXT file and then counted using the primitive \texttt{wc} UNIX command line utility, which includes all titles, page numbers, captions, citations, equations and other non-content-characters into the character count it outputs. Therefore, it can be said that, were a proper character count possible, it is quite probable that it would approach if not fall below the 60000 character limit. Moreover, it should be mentioned that this thesis initially had over 180000 characters and was thus already reduced dramatically. It is likely that any further deletion of content would hinder the reader's understanding of the topics presented. \parbreak \parbreak \parbreak \parbreak

{ \centering

Villach, \today, \parbreak \parbreak \parbreak

\includegraphics[scale=0.18]{../img/signature} \parbreak

(Peter Goldsborough) \parbreak \parbreak \parbreak

}

\pagebreak

{ \small
\begin{tabular}{| p{2cm} | p{13cm} |}
  \hline
  \rowcolor[gray]{0.8}
  Date & Event \\
  \hline
  15.11.2013 &  Finding, formulation and narrowing down of thesis topic. Determined to be related to the creation of a digital synthesizer. At this point in German language.\\
  \hline
  01.12.2013 + 04.12.2013 & E-mail exchange with Martin Kastner. Discussion and revision of the thesis concept. \\
  \hline
  01.01.2014 & Start of C++ synthesizer project. \\
  \hline
  09.01.2014 & E-mail exchange with John Cooper. Discussion of Wavetables and optimal Wavetable size. \\
  \hline
  13.01.2014 & E-mail exchange with John Cooper. Discussion of Wavetable generation methods, optimal Wavetable size and interpolation algorithms. \\
  \hline
  17.1.2014 & E-mail exchange with John Cooper. Discussion of Wavetable size, comparison with direct calculation, performance considerations. \\
  \hline
  31.1.2014 & E-mail exchange with John Cooper. Discussion of WAVE files and implementation. \\
  \hline
  05.02.2014 & E-mail exchange with Martin Kastner. Clarification about citing source code. \\
  \hline
  05.02.2014 & E-mail exchange with Jari Kleimola. Sharing of available DSP and audio programming resources online. Introduction of the concept of Wavetables. \\
  \hline
  18.02.2014 & E-mail exchange with John Cooper. Explanation of current inheritance pattern in implementation. Discussion of WAVE file volume considerations. \\
  \hline
  10.03.2014 & E-mail exchange with Jari Kleimola. Discussion of Wavetable algorithms, optimal Wavetable size, Frequency Modulation Synthesis. \\
  \hline
  20.05.2014 & E-mail exchange with John Cooper. Advice to use the Resource-Acquisition-Is-Initializatoin (RAII) concept and use objects to store pointers rather than working with loose pointers. \\
  \hline
  28.05.2014 & Physical Meeting with Martin Kastner. Finalized decision to switch the thesis' language --- from German to English. \\
  \hline
  10.09.2014 & Physical Meeting with Martin Kastner. Update on progress, process and methods. \\
  \hline
  17.09.2014 & Physical meeting with Martin Kastner. Discussion of problems concerning the thesis and synthesizer project. \\
  \hline
  17.12.2014 & Physical meeting with Martin Kastner. Discussion of thesis-related information, such as: APA citing rules, citing rules for online website sources, for online forum sources, for online image sources. Considerations for formatting and content. \\
  \hline
  20.12.2014 & Start of thesis \LaTeX project. Wrote first chapter --- From Analog To Digital. \\
  \hline
  22.12.2014 & Finished second chapter --- Generating Sound. \\
  \hline
  24.12.2014 & Finished third chapter --- Modulating Sound. \\
  \hline
  25.12.2014 & Finished fourth chapter --- Crossfading Sound. \\
  \hline
  27.12.2014 & Finished fifth chapter --- Filtering Sound. \\
  \hline
  29.12.2014 & Finished sixth chapter --- Effects. \\
  \hline
  30.12.2014 & Finished seventh chapter --- Synthesizing Sound. \\
  \hline
  02.01.2015 & Finished eighth chapter --- Recording Sound --- and ninth chapter --- Sound output. \\
  \hline
  04.01.2015 & Edited and revised thesis. Final character count of extended version: 181319 characters. Fork of current thesis project to new project for shortened version. \\
  \hline
  12.01.2015 & Removed chapters 4, 6 and 9. Wrote Introduction. \\
  \hline
  18.01.2015 & Removed chapters 5 and 8. Left with chapters 1, 2, 3 and 7. Wrote Conclusion. \\
  \hline
  21.01.2015 & Wrote abstract; edited, revised, formatted. \\
  \hline
  25.01.2015 & Final edits and format improvements. \\
  \hline
\end{tabular}

}
\end{document}
